\documentclass[10pt]{article}
\usepackage[margin=26mm]{geometry}
\usepackage{amsthm}
\usepackage{amsmath}
\usepackage{mathbbol}
\usepackage{amsfonts}
\usepackage{amssymb}
\usepackage{graphicx}
\usepackage{algorithm}
\usepackage{url}
\usepackage{color}
\usepackage{todonotes}

\usepackage{lineno}
%\linenumbers

\usepackage{subfig}

\usepackage{algorithm}
\usepackage{algpseudocode}
\usepackage{comment}

% \usepackage{ccaption}

\newcommand{\N}{\mathbb{N}}
\newcommand{\Q}{\mathbb{Q}}
\newcommand{\R}{\mathbb{R}}
\newcommand{\Z}{\mathbb{Z}}
\newcommand{\eps}{\epsilon}
\newcommand{\Nrv}{\mathrm{Nrv}}

\newcommand{\ignore}[1]{}

\newtheorem{theorem}{Theorem}
\newtheorem{lemma}[theorem]{Lemma}
\newtheorem{proposition}[theorem]{Proposition}
\newtheorem{corollary}[theorem]{Corollary}
\newtheorem{definition}[theorem]{Definition}
\newtheorem{remark}[theorem]{Remark}
\newtheorem{note}[theorem]{Note}

\def\marrow{\marginpar[\hfill$\longrightarrow$]{$\longleftarrow$}}
\def\michael#1{\textcolor{red}{\textsc{Michael says: }{\marrow\sf #1}}}
\def\mike#1{\textcolor{red}{\textsc{Mike says: }{\marrow\sf #1}}}

\title{scc2020: A File Format for Sparse Chain Complexes in TDA}
%
\author{
Michael Lesnick, Michael Kerber
}
\date{\today}


\begin{document}

\maketitle
\begin{abstract}
Several software projects concerning efficient computations for multiparameter persistence are now underway.  With this, the need for a standard file format for chain complexes and presentations of multiparameter persistence modules has become apparent: Having the various codes adhere to such a standard would allow us to more easily compare the performance of software and leverage the strengths of multiple software packages in our data analyses.  This document proposes such a standard. 
\end{abstract}
%
\section{Setup}
Our goal is a succinct representation of chain complexes of the form
%
\begin{eqnarray}
M_1\xrightarrow{f_1} M_2\xrightarrow{f_2} M_3 \xrightarrow{f_3} \ldots \xrightarrow{f_{n-1}} M_n\xrightarrow{f_n}  M_{n+1}
 \label{cc}
\end{eqnarray}
where $M_1,\ldots,M_{n+1}$ are finitely generated free $d$-parameter persistence modules with base field $K$
as defined in~\cite[p.18]{lw-multi}, and $f_1,\ldots,f_n$ are homomorphisms
between these modules. 
Every module $M_i$
is determined by an (ordered) set of homogeneuous generators;
write $m_i$ for the number of generators, $G_{i,1},\ldots,G_{i,m_i}$
for the generators and $g_{i,1},\ldots,g_{i,m_i}$ for their grades in $\R^d$.
Then, every $f_i$ can be represented as a $m_{i+1}\times m_i$-matrix with entries in $K$, where the $j$-th column represents $f_i(G_{i,j})$
as a linear combination (with shifts) of the generators $G_{i,1},\ldots,G_{i,m_i}$

\begin{remark}
An important special case is where $n=2$, in which case the chain complex is simply a presentation of the cokernel of $f_1$.
\end{remark}

\begin{remark}
Chain complexes of non-free persistence modules do arise in TDA, from \emph{multi-critical} multi-filtrations.  In the future it might be worthwhile to extend this specification to handle these.
\end{remark}


\section{Format}

The format allows to include an arbitrary number of empty lines
and lines starting with \#, which are just ignored.

\begin{itemize}

\item We require that the first (non-empty, non-commented) line of the file contains only the word ''scc2020''.  

\item The second line contains an integer $d\geq 0$ which denotes the number of persistence parameters.

\item The third line contains a sequence of $n+1$ integers $m_1,..,m_{d+1}$, 
separated by spaces, denoting the size of the generating sets 
of $M_1,..,M_{d+1}$.

\item The next line, which is optional, is of the form ``-{}-reverse [list]'', where [list] is a sequence of numbers between $1$ and $d$ separated by spaces.  If the number $j$ appears in this list, then the multiparameter persistence modules will be taken to have decreasing coordinate direction $j$.  For example, if $d=2$, then ``-{}-reverse 1'' indicates that we are working with $\R^{\mathrm{op}}\times \R$-indexed persistence modules, whereas if this line is omitted, then this indicates that we are working with $\R^2$-indexed persistence modules.  To avoid confusion and bugs, the line ``-{}-reverse'' with no arguments should return  an error.

\item The file continues with $d$ blocks consisting of $m_1,m_2,...,m_d$ lines,
each block specifying the grades of the generators of $M_i$ 
and the columns of the matrix for $f_i$.
Each line in a block has the form

\begin{quote}
$v_1$ $v_2$ $\ldots$ $v_d$ [;] $k_1$[C$x_1$] $k_2$[C$x_2$] $\ldots$ $k_m$[C$x_m$]
\end{quote}

For the $j$-th line in block $i$, $v_1,\ldots,v_d$ determines the grade 
of the $j$-th basis element of $M_i$. 
After the (optional) semicolon, the image of the basis element 
under $f_i$ is specified by a sequence of words of the form $k$C$x$ 
with $k$ an integer and $x$ specifying a field element.  This fixes 
the coefficient of the $(k,j)$-entry in the matrix to be $x$. 
The values $k_1,\ldots,k_m$ do not have to be in increasing order, 
but no repetitions are allowed.  Unspecified row entries are interpreted to be $0$.  Moreover, one can also just specify $k$ (without C$x$) which is interpreted as $k$C$1$.%\todo{perhaps specify that commas can also be used in place of spaces?  But this is a little awkward with the semicolon convention, unless you specify that multiple characters will be treated as a single character.}
\item Finally, an optional block of $m_{d+1}$ lines can be added, specifying
the grades of the generating set for $M_{d+1}$. 
If used, the lines are of the form
\begin{quote}
$v_1$ $v_2$ $\ldots$ $v_d$
\end{quote}
as above.  The reason this block is optional is that if one is only interested in the homology modules \[\ker f_{i+1}/\mathrm{im} f_i\] for $i\in \{2,\ldots, n\},$ then this block is not needed.

\end{itemize}

\paragraph{Field elements and grades}
%
This proposal does not specify the format of the grades ($v_j$ above) 
or of the field elements ($x_j$ above), though a future revision may specify this. For the grade coordinates, 
it is recommended to use a string with an obvious representation as a real
number, for instance a decimal number or perhaps fraction.
For the field elements, it is recommended to intepret integers in the obvious
way (i.e., $1$ is the one in the field, $3 = 1+1+1$, and so on).
If chain complexes with more exotic field coefficients are specified,
we recommend to describe them as a comment in the file.
%\todo{Perhaps to make things concrete, we should accept decimals, scientific notation, and fractions for grades, and just integers for fields (assume the field is prime).  No other fields are getting used in TDA.}


\section{Examples}

We demonstrate which data can be represented in the defined format.

\paragraph{Simplicial complexes.}
%
A filtered simplicial complex defines a chain complex in a natural way,
where the morphisms are defined by the boundary operator.
As an example, we consider the Vietoris-Rips filtration
for the three points $A=(0,0)$, $B=(3,0)$, and $C=(0,4)$.
The full complex consists of the $3$ vertices, 
$3$ edges and one triangle.  %which we
%orient as denoted in Figure~\ref{fig:??}.
We obtain the sequence
\[
M_1\xrightarrow{f_1} M_2\xrightarrow{f_2} M_3
\]
where $M_1$ is generated by $ABC$, $M_2$ by $AB,AC,BC$ and
$M_3$ by $A,B,C$. 
By definition of the Vietoris-Rips filtration,
the three vertices are graded by $0$.
Since the length of the three edges are $3$, $5$, and $4$,
the edges are graded by $1.5$, $2.5$, and $2$, and the triangle
by $2.5$. Hence, we can represent the simplicial complex in ssc2020 format as
\begin{verbatim}
#Lines starting with # are comments
ssc2020
#Number of parameters
1
#Sizes of generating sets
1 3 3
#First block
2.5 ; 1C1 2C-1 3C1
#Second block
1.5 ; 1C-1 2C1
2.5 ; 1C-1 3C1
2   ; 2C-1 3C1
#Third block (optional)
0
0
0
\end{verbatim} 

If the complex is only used for computations over $\Z_2$ coefficients
(which is an important case in many applications), we can skip orientations
and simplify further:

\begin{verbatim}
ssc2020
1
1 3 3

2.5 ; 1 2 3

1.5 ; 1 2
2.5 ; 1 3
2   ; 2 3

0
0
0
\end{verbatim} 

\paragraph{Free implicit representations}
%
This data format is inspired by the firep data structure in Rivet~\cite[Sec 5.1]{lw-multi}; see \url{https://rivet.readthedocs.io/en/latest/inputdata.html#firep-algebraic-input}.  There are in fact two slightly different firep formats introduced in RIVET.  The latter is a in a soon-to-be-merged pull request and is intended to replace the old one, though RIVET will be backwards compatible.  

A file in the new firep format can be converted directly into the ssc2020 format in the following way:
\begin{itemize}
\item[1.] Include the line ``ssc2020'' at the top (and remove the line ``-{}-datatype firep'', if present),
\item Insert a line containing just the character 
$2$ after it. (This specifies that one considers a 2-parameter persistence module.)
\item If the firep file contains the lines ``--xreverse'' or ``--yreverse'', these should be deleted, and the line ``-{}-reverse [list]'', with list taken to be be the appropriate thing, should be given on the fourth line, as explained above.
\item Other optional command-line flags, beginning with ``-{}-'', may need to be removed from the file, if present.
\end{itemize}
Converting a file in the old firep format to our format is similar.

\bibliographystyle{plain}
\bibliography{bib}


\end{document}